\subsection{Stetigkeit}

\begin{Definition}{Stetigkeit (Continuity)}{}
    Funktion ist stetig:
    \begin{equation*}
     \iff \lim_{x\rightarrow x_0} f(x) = f(x_0) \text{  [Normale Definition]} \iff \text{Jedes $f_i$ stetig} \text{(Intuition)}\\
    \end{equation*}
    Definition Stetigkeit mit Folgen: Für jede Folge $(x_n)$ sodass $x_n \rightarrow x$ für $x\rightarrow \infty $:
    \[
    (f(x_n)) \rightarrow f(x)
    \]
\end{Definition}

Sei $f,g$ stetig: $f+g$, $f\cdot g$, $\frac{f}{g}$ (wenn g nie 0 ist), $f \circ g$ stetig.

Falls $f$ stetig, gilt

\[
    \lim_{x \rightarrow a} f(x) = f(\lim_{x\rightarrow a} x)
\]

$f$ diffbar $\Rightarrow$ $f$ stetig $\Rightarrow$ $f$ integrierbar

$f$ nicht integrierbar $\Rightarrow$ $f$ nicht stetig $\Rightarrow$ $f$ nicht diffbar.

\begin{Rezept}{Polarkoordinatentrick (Change of Variable, Coordinates)}{}
    Ziel: Zeige oder widerlege Stetigkeit. Seien $x=r\cos \varphi$, $y=r\sin \varphi$. Berechne
    \[
    \lim_{(x, y) \rightarrow (0,0)} f(x, y) = \lim_{r \rightarrow 0} f(x, y)
    \]
    Hängt das Resultat von $\varphi$ ab $\Rightarrow$ der Grenzwert existiert nicht $\Rightarrow$ nicht stetig an dieser Stelle.
\end{Rezept}

\begin{Rezept}{Linientrick}{}
    Ziel: Stetigkeit widerlegen. Suche zwei Linien, die einen unterschiedlichen $\lim$ haben. Zeigt, dass ein $\lim$ nicht existieren kann.
    Sei $f(x, y)=\frac{y}{x+1}$ und $\{(x, y) \in \R \mid x \neq 1\}$ für $(x, y) \rightarrow (-1, 0)$. Linie $\{(x, y) \in \R \mid y=0\cap x \neq 1\}=0$ und $\{(x, y) \in \R \mid y=x+1\}=1$.\\
    
    \textbf{Bemerkung:} Meistens à la: $f(x,y)$ verschwindet auf der Linie $\{x=...\}$, dann müsste $\lim_{(x,y)\rightarrow(0,0)} f(x,y) = 0$, aber für $x=...y...$ sehen wir, dass
    $f(x,y) =$ fancy Expression = $1 \neq 0$. Da $x=...$ auf der Linie liegt, folgt der Widerspruch. \textbf{Todo: Quasi Ausnahme finden! intuitiv erläutern!}
\end{Rezept}

\begin{Beispiel}{Linientrick}{}
\[ \text{Gegeben } f(x,y) = \frac{y}{x-1} ~~ \text{existiert} ~~ \lim_{(x,y)\rightarrow(1,0)} f(x,y) ~~ \text{?}\]
\\
\textbf{Lösung}: Wir sehen $f$ auf der Linie $\{ (x,y) | y=0, \; x \neq 1 \}$ verschwindet. Hätte also
$f$ einen Grenzwert für $(x,y) \rightarrow (1,0)$ wäre dieser gleich $0$. Aber die Linie $y=x-1$ geht
durch $(1, 0)$ und auf dieser Linie ist $f$ gleich $1$. Also existiert $\lim_{(x,y) \rightarrow (1, 0)} f(x,y)$ nicht.
\end{Beispiel}

\begin{Beispiel}{Vergleichstrick / Sandwich}{}
\[ \text{Gegeben } f(x,y) = \frac{(x-1)^2 \ln(x)}{(x-1)^2 + y^2} ~~ \text{existiert} ~~ \lim_{(x,y)\rightarrow(1,0)} f(x,y) ~~ \text{?}\]
\\
\textbf{Lösung}: 
\[ |f(x,y)| = \frac{|(x-1)^2|}{|(x-1)^2 + y^2|} \; |ln(x)| ~~ \leq |ln(x)| \]
Wegen $\lim_{x\rightarrow 1} ln(x) = 0$ sehen wir, dass 
\[ 0 \leq \lim_{(x,y) \rightarrow (1,0)} |f(x,y) \leq |\lim_{x\rightarrow 1} |\ln(x)| = 0\]
und damit ist auch der gesuchte Grenzwert $ = 0$.
\end{Beispiel}


\begin{Rezept}{Stetigkeit prüfen}{}
    Sei $f$ die zu prüfende Funktion. 1) $f$ muss überall definiert sein. 2) $\lim_{x \rightarrow a} f(x)$ existiert. 3) $\lim_{x \rightarrow a} f(x) = f(a)$.
\end{Rezept}

\begin{Diverses}{Wichtige Fakten}{}
Wenn $f: \mathbb{R}^n \rightarrow \mathbb{R}$ stetig ist, dann ist für alle $a \leq b$
\[
\{x \in \R^n | a \leq f(x) \leq b\} \subset \mathbb{R}^n 
\]
geschlossen. \\
Wenn  $X \subset \mathbb{R}^n$ kompakt und $f: X \rightarrow \mathbb{R}$ stetig, dann hat f mindestens ein \textbf{Maximum} und mindestens ein \textbf{Minimum}.
\end{Diverses}
