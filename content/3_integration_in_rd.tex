\section{Integration in $\mathbb{R}^d$}

\subsection{Wegintegrale}

Sei $f: [a, b] \rightarrow \mathbb{R}^{n}$ stetig, d.h. für
\[ f(t) = (f_1(t), ~ \ldots, ~ f_n(t)) \]
jedes $f_i$ stetig, dann ist
\[ \int_a^b f(t) dt = \left( \int_a^b f_1(t) dt, ~ \ldots , ~ \int_a^b f_n(t) dt\right) \in \mathbb{R}^n \]

Für eine parametrisierte Kurve in $\mathbb{R}^n$, d.h.
$\gamma : [a, b] \rightarrow \mathbb{R}^n$, s.d.
\begin{enumerate}
\item{ $\gamma$ stetig}
\item{2. $\exists t_0, \ldots, t_k$, ~ s.d. $t_0 = a < t_i < t_k = b$, ~~ s.d. ~
$\gamma ~ | ~ ]t_i, t_{i-1}[ ~ \in C^1$}
\end{enumerate}
nennen wir $\gamma$ einen Pfad zwischen
$\gamma(a)$ und $\gamma(b)$. 

\begin{Satz}{Länge einer Kurve}{}
	Sei $\gamma$ eine reguläre Kurve $t \to \gamma(t)$ sei $|.|$ die euklidische Norm: Die Länge ist
	\[
  		L(\gamma) = \int_a^b |\gamma'| dt
  	\]
\end{Satz}

\begin{Rezept}{Wegintegrale}{}
	Gegeben: Vektorfeld f von der Klasse $C^1$ und eine Kurve $\gamma \in C^1_{pw}$.
	
	Gesucht: Wegintegral $\int_\gamma f \cdot ds$.\\
	
	\textbf{Lösungsschritt I:}
	
	Parametrisiere $\gamma$, d.h. finde eine Abbildung $\gamma(t) : [a, b] \to \R^n$, $t \to \gamma(t)$.
	
	\textbf{Lösungsschritt II:}
	
	Berechne $\gamma'(t) = \frac{d}{dt} \gamma(t)$. Dabei wird jede Komponente des Vektors $\gamma$ einzeln nach t abgeleitet.
	
	\textbf{Lösungsschritt III:}
	
	Das Wegintegral von $f$ entlang $\gamma$ ist definiert als
	\[
		\int_{\gamma} f(s)\cdot d\vec{s} = 
		\int_a^b \underbrace{\underbrace{f(\gamma(t))}_{\in\mathbb{R}^n}  \cdot 
		\underbrace{\gamma'(t)}_{\in\mathbb{R}^n}}_{\text{Skalarprodukt in } \mathbb{R}} dt  ~
		\in \mathbb{R}
	\]
	und ist unabhängig der gewählten Parametrisierung!
\end{Rezept}

\subsection{Potential}

Intuition: Stärke der Änderung der Richtung der Vektoren in einem Vektorfeld.

\begin{Definition}{Potentialfelder und Potentiale}{}
	Ein Vektorfeld $\vec{v} : \Omega \subset \R^n \to \R^n$ heisst \textbf{Potentialfeld}, falls eine stetig differenzierbare Abbildung $\Phi : \Omega \subset \R^n \to \R$ existiert, sodass \[\vec{v} = \nabla \Phi\] gilt. Das skalare Feld $\Phi$ heisst dann \textbf{Potential} von $\vec{v}$. \textbf{Wichtig:} Es gibt sehr viele Vektorfelder, die sich nicht als Gradient eines skalaren Feldes schreiben lassen (also keine Potentialfelder sind)!
\end{Definition}

\subsection{Konservative Vektorfelder (= Potentialfelder)}
Ein Vektorfeld $V: (x, y) \mapsto (f_1(x), f_2(y))$ ist konservativ (= ein Potentialfeld) wenn es überall definiert
und zusammenhängend (für 2D: keine 'Löcher') ist und gilt, dass $rot \;V = 0$. Es gelten die \textbf{Integrabilitätsbedingungen} \\ 
Für $\mathbb{R}^2$:
\[ \frac{\partial f_1}{\partial x_2} = \frac{\partial f_2}{\partial x_1} \]
Für $\mathbb{R}^3$:
\[ \frac{\partial f_1}{\partial x_2} =  \frac{\partial f_2}{\partial x_1}, 
~~  \frac{\partial f_1}{\partial x_3} = \frac{\partial f_3}{\partial x_1},
~~ \frac{\partial f_2}{\partial x_3} =  \frac{\partial f_3}{\partial x_2}
\]
Für $\mathbb{R}^n$:
\[
	\frac{\partial f_i}{\partial x_j} =  \frac{\partial f_j}{\partial x_i},
	\quad
	\forall i \neq j,
	\quad
	i, j \in \{1,...,n\}
\]

\begin{Satz}{Wegintegrale für Potentialfelder}{}
	Sei $\vec{v} : \Omega \subset \R^n \to \R^n$ ein Potentialfeld mit Potential $\Phi$. Dann gilt für jedes Wegintegrale entlang $\gamma$, dass
	\[
		\int_\gamma \vec{v} \cdot d\vec{s} = 
		\int_a^b \vec{v}(\gamma(t)) \cdot \gamma'(t) dt =
		\Phi(\gamma(b)) - \Phi(\gamma(a))
	\]
	Wir müssen also nur die Potentiale am Anfangs- und Endpunkt der Kurve auswerten! Damit sieht man auch gerade, dass für jede geschlossene Kurve das Wegintegrale eines Potentialfeldes gleich 0 ist. 
\end{Satz}

\begin{Diverses}{Zusammenfassung}{}
	Sei $\vec{v} : \Omega \subset \R^n \to \R^n$ ein stetig differenzierbares Vektorfeld und $\Omega$ einfach zusammenhängend. Folgende Aussagen sind äquivalent:
	\begin{itemize}
		\item $\vec{v}$ ist konservatives Vektorfeld
		\item $\vec{v}$ ist ein Potentialfeld
		\item Für alle geschlossene Kurven gilt $\oint \vec{v} \cdot d\vec{s} = 0$
		\item Das Integral $\int_\gamma \vec{v} \cdot d\vec{s}$ ist unabhängig vom Weg
		\item $\vec{v}$ erfüllt die Integrabilitätsbedingung auf $\Omega$. Für $\R^3$ gilt also $\nabla \times \vec{v} = 0$
	\end{itemize}
\end{Diverses}
\subsection{Sternförmig?}
Zügs zu Star shaped?

\begin{Satz}{Satz von Fubini}{}
    Reduzierung von mehrdimensionalen Integralen auf eine Dimension. Sei $f: [a,b] \times [c, d]$ stetig, dann gilt
    \[ \int_a^b \int_c^d f(x, y) dx ~ dy = \int_c^d \int_a^b f(x, y) dy ~ dx = \int_{[a,b] \times [c, d]} f(x, y) dx ~ dy   \]

    Es sei der Quader $Q = [a_1,a_2] \times [a_2, b_2] \times \dots \times [a_n, b_n]$ mit $f \in C^0(Q)$ gegeben. Dann gilt

    \[
        \int_Q f(x) d\mu(x) = \int_{a_1}^{b_1} dx_1 \int_{a_2}^{b_2} dx_2 \dots \int_{a_n}^{b_n} dx_n f(x_1, x_2,...,x_n)
    \]
\end{Satz}
Die Integrationsreihenfolge darf vertauscht werden.\\

Alternative Schreibweise von $dx ~ dy$: $\mu(x, y)$, $\mu(x+y)$.

\subsection{Normalbereiche in $\mathbb{R}^2$}

Sei $\Omega$ eine beschränkte Teilmenge von $\mathbb{R}^2$. $\Omega$ heisst \textbf{y-Normalbereich}, falls sich $\Omega$ wie folgt darstellen lässt:

\[
    \Omega = \{(x, y) \in \mathbb{R}^2 \mid a \leq x \leq b, f(x) \leq y \leq g(x)\}
\]

wobei $f$, $g$ stetige Funktionen der Variable $x$ sind. Die Rolle von $x$ und $y$ darf vertauscht werden (es existiert also auch ein $x$-Normalbereich).

Sei \[\Omega = \{(x, y) \in \mathbb{R}^2 \mid a \leq x \leq b, f(x) \leq y \leq g(x)\}\] ein Normalbereich mit stetigen Funktionen $f$, $g$ und sei die zu integrierende Funktion $F \in C^0(\Omega)$, dann gilt:

\[
    \int_{\Omega} F d\mu = \int_a^b dx \int_{f(x)}^{g(x)} dy F(x, y)
\]

Das innere Integral wird zuerst ausgewertet.

\begin{Satz}{Substitutionsregeln in einer Dimension}{}
    Sei $f$ eine Riemann-integrierbare Funktion. Für die Berechnung des Integrals
    \[
        \int_a^b f(x) dx
    \]
    führt die Substitution $x \to g(u)$ zu $dx = g'(u)du$ und damit wird das Integral
    \[
        \int_a^b f(x) dx = \int_{g^{-1}(a)}^{g^{-1}(b)} f(g(u)) g'(u) du
    \]
    Das heisst wir haben das Integrationselement $dx$ durch $g'(u)du$ ersetzt und die Grenzen entsprechend angepasst.
\end{Satz}

\begin{Satz}{Substitutionsregeln in $n$ Dimensionen}{}
    Sei $f$ eine Riemann-integrierbare Funktion auf dem Gebit $\Omega \subset \R^n$ und die Koordinatentransformation (Substitution)
    \[
    (x_1,\hdots,x_n) = \Phi(u_1, \hdots,  u_n)
    \]
    oder in Komponenten
    \[
        \vek{x_1}{\vdots}{x_n}
        = \Phi(u)
        = \vek{g_1(u_1,\hdots,u_n}{\vdots}{g_n(u_1,\hdots,u_n)}
    \]
    ist ein $C^1$-Diffeomorphismus. Dann gilt
    \[
        \int_\Omega f(x_1, \hdots, x_n)dx_1\hdots dx_1 = \int_{\widetilde{\Omega}} f(g_1(u), \hdots, g_n(u))|\det d \Phi | du_1\hdots du_n
    \]
    wobei das Gebiet $\widetilde{\Omega} = \Phi^{-1}(\Omega)$ ist. $|det d\Phi|$ ist die \textbf{Funktionaldeterminante}.
\end{Satz}

\begin{Rechenregeln}{Wichtige Koordinatentransformationen und Funktionaldeterminanten}{}
    \begin{itemize}
       \item Polarkoordinaten in $\R^2$ \begin{alignat*}{4}
            x &= r \cos \varphi\quad &&0 \leq r < \infty\quad &&&dxdy = r \cdot drd\varphi\\
            y &= r \sin \varphi\quad &&0 \leq \varphi < 2\pi\quad
        \end{alignat*}
        \item Elliptische Koordinaten $\R^2$ \begin{alignat*}{4}
            x &= r \cdot  a \cos \varphi\quad &&0 \leq r < \infty\quad &&&dxdy = a \cdot b \cdot  r \cdot drd\varphi\\
            y &= r \cdot b \sin \varphi\quad &&0 \leq \varphi < 2\pi\quad
        \end{alignat*}
        \item Zylinderkoordinaten $\R^3$ \begin{alignat*}{4}
            x &= r \cdot  a \cos \varphi\quad &&0 \leq r < \infty\quad &&&dxdydz = r \cdot drd\varphi dz\\
            y &= r \cdot b \sin \varphi\quad &&0 \leq \varphi < 2\pi\quad\\
            z &= z\quad &&\-\infty \leq z < \infty\quad
        \end{alignat*}
        \item Kugelkoordinaten $\R^3$ \begin{alignat*}{4}
            x &= r \cdot \sin \theta \cos \varphi \quad &&0 \leq r < \infty\quad &&&dxdydz = r^2 \sin \theta \cdot drd\theta d\varphi\\
            y &= r \cdot \sin \theta \sin \varphi \quad && 0 \leq \theta < \pi\quad\\
            z &= r \cos \theta \quad &&0 \leq \varphi < 2\pi\quad\quad
        \end{alignat*}
   \end{itemize}
\end{Rechenregeln}