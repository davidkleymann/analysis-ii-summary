\subsection{Lagrange Multiplikatoren}
Sei $f(x) \in \R^{n}$ die zu maximierende Funktion, von der wir aber nur Punkte
betrachten wollen, für welche gilt, dass $g(x) = 0$ mit $g(x) \in \R^{l}$
Definition der \textbf{Lagrange-Funktion}:
\[ L = f-\lambda \cdot g = f - \lambda_{1} g_{1} - \ldots - \lambda_{n}g_{n} ~ \text{ mit $\lambda$ in } \R^{l} \]
Dieses $\lambda$ existiert immer, wenn $f, g \in C^{1}$.
Die Kandidaten für Extrema von $f$ unter der Nebenbedingung $g = 0$ sind genau die
kritischen Punkte der Lagrange-Funktion $L$. Mittels der Hesse-Matrix von $L$
kann die Art der Extrema gefunden werden. Jeder Kandidat wird in $f$ eingesetzt
um zu erkennen, wo $f$ mit welchen Werten Extrema annimmt.

\begin{Rezept}{Extremwertaufgaben mit Nebenbedingungen}{}
	Gegeben: $f:\Omega\subset \R^n \to \R$ und $g:\Omega \to \R^l$ der Klasse $C^1$.\\
	Gesucht: ein Extremum der Funktion $f$ unter der Nebenbedingung $g=0$.\\
	\newline
	\textbf{Lösungsschritt I:}\\
	Bilde die Lagrange-Funktion
	\begin{equation*}
	L(x_1,\dots,x_n,\lambda) = f(x_1,\dots,x_n) - \lambda_1 g_1(x_1,\dots,x_n) - \dots - \lambda_l g_l(x_1,\dots,x_n),
	\end{equation*}
	wobei $g_1,\dots,g_l$ die $l$ Komponenten von $g$ sind. Oft ist $l=1$.\\
	\newline
	\textbf{Lösungsschritt II:}\\
	Bestimme die kriischen Punkte von L, d.h. löse
	\begin{align*}
	\pdv{L}{x_1} &= \pdv{f}{x_1}-\lambda_1\pdv{g_1}{x_1} - \dots -\lambda_l\pdv{g_l}{x_1} = 0\\
	\vdots\\
	\pdv{L}{x_n} &= \pdv{f}{x_n}-\lambda_1\pdv{g_1}{x_n} - \dots -\lambda_l\pdv{g_l}{x_n} = 0.
	\end{align*}
	
	Löse dieses Gleichungssystem mit den zusätzlichen Gleichungen von $g=0$.\\
	\newline
	\textbf{Lösungsschritt III:}\\
	Die Lösungen $(x_1,\dots,x_n)$ sind die Kandidaten für Extremalstellen von $f$. Untersuche die Hesse-Matrix von L um den Typ zu bestimmen.
	\begin{align*}
	\text{Hess}(L)(x_0,\lambda)\text{ ist positiv definit} \quad &\rightarrow  \quad 
	x_0 \text{ ist ein lokales Minimum von } f \text{ auf } g^{-1}(0),\\
	\text{Hess}(L)(x_0,\lambda)\text{ ist negativ definit} \quad &\rightarrow  \quad x_0 \text{ ist ein lokales Maximum von } f \text{ auf } g^{-1}(0),\\
	\text{Hess}(L)(x_0,\lambda)\text{ ist indefinit} \quad &\rightarrow  \quad x_0 \text{ ist ein Sattelpunkt von } f \text{ auf } g^{-1}(0).
	\end{align*}
	
	Ist man nur an den \textbf{globalen Extremwerten} interessiert, spart man sich diesen Schritt und wertet stattdessen die Funktion $f$ an den Kandidaten aus.\\
	\newline
	\textbf{Lösungsschritt IV:}\\
	Ist man auch an Extremalstellen im Innern $\overset{\circ}{\Omega}$ interessiert, sucht man diese mit dem bereits bekannten Rezept.
\end{Rezept}

TODO: falls no platz: beispiele für extremwertaufgaben mit nebenbedingungen breckman (?)