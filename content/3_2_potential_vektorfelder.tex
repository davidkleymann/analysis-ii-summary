\subsection{Potential}

Intuition: Stärke der Änderung der Richtung der Vektoren in einem Vektorfeld.

\begin{Definition}{Potentialfelder und Potentiale}{}
	Ein Vektorfeld $\vec{v} : \Omega \subset \R^n \to \R^n$ heisst \textbf{Potentialfeld}, falls eine stetig differenzierbare Abbildung $\Phi : \Omega \subset \R^n \to \R$ existiert, sodass \[\vec{v} = \nabla \Phi\] gilt. Das skalare Feld $\Phi$ heisst dann \textbf{Potential} von $\vec{v}$. \textbf{Wichtig:} Es gibt sehr viele Vektorfelder, die sich nicht als Gradient eines skalaren Feldes schreiben lassen (also keine Potentialfelder sind)!
\end{Definition}

\subsection{Konservative Vektorfelder (= Potentialfelder)}
Ein Vektorfeld $V: (x, y) \mapsto (f_1(x), f_2(y))$ ist konservativ (= ein Potentialfeld) wenn es überall definiert
und zusammenhängend (für 2D: keine 'Löcher') ist und gilt, dass $rot \;V = 0$. Es gelten die \textbf{Integrabilitätsbedingungen} \\ 
Für $\mathbb{R}^2$:
\[ \frac{\partial f_1}{\partial x_2} = \frac{\partial f_2}{\partial x_1} \]
Für $\mathbb{R}^3$:
\[ \frac{\partial f_1}{\partial x_2} =  \frac{\partial f_2}{\partial x_1}, 
~~  \frac{\partial f_1}{\partial x_3} = \frac{\partial f_3}{\partial x_1},
~~ \frac{\partial f_2}{\partial x_3} =  \frac{\partial f_3}{\partial x_2}
\]
Für $\mathbb{R}^n$:
\[
	\frac{\partial f_i}{\partial x_j} =  \frac{\partial f_j}{\partial x_i},
	\quad
	\forall i \neq j,
	\quad
	i, j \in \{1,...,n\}
\]

\begin{Satz}{Wegintegrale für Potentialfelder}{}
	Sei $\vec{v} : \Omega \subset \R^n \to \R^n$ ein Potentialfeld mit Potential $\Phi$. Dann gilt für jedes Wegintegrale entlang $\gamma$, dass
	\[
		\int_\gamma \vec{v} \cdot d\vec{s} = 
		\int_a^b \vec{v}(\gamma(t)) \cdot \gamma'(t) dt =
		\Phi(\gamma(b)) - \Phi(\gamma(a))
	\]
	Wir müssen also nur die Potentiale am Anfangs- und Endpunkt der Kurve auswerten! Damit sieht man auch gerade, dass für jede geschlossene Kurve das Wegintegrale eines Potentialfeldes gleich 0 ist. 
\end{Satz}

\begin{Diverses}{Zusammenfassung}{}
	Sei $\vec{v} : \Omega \subset \R^n \to \R^n$ ein stetig differenzierbares Vektorfeld und $\Omega$ einfach zusammenhängend. Folgende Aussagen sind äquivalent:
	\begin{itemize}
		\item $\vec{v}$ ist konservatives Vektorfeld
		\item $\vec{v}$ ist ein Potentialfeld
		\item Für alle geschlossene Kurven gilt $\oint \vec{v} \cdot d\vec{s} = 0$
		\item Das Integral $\int_\gamma \vec{v} \cdot d\vec{s}$ ist unabhängig vom Weg
		\item $\vec{v}$ erfüllt die Integrabilitätsbedingung auf $\Omega$. Für $\R^3$ gilt also $\nabla \times \vec{v} = 0$
	\end{itemize}
\end{Diverses}
