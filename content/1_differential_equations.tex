\section{Differentialgleichungen}

\begin{Definition}{Differentialgleichung}{}
    Eine \textbf{Differentialgleichung} ist eine Gleichung, in welcher eine unbekannte Funktion $y(x)$ einer oder mehreren Variablen und ihre Ableitung vorkommnt. Im Falle, dass $y$ eine auf dem Intervall $I \subset \R$ definierte Funktion ist \[y: I \subset \R \to \R,\] und die Funktion und ihre Ableitungen alle an der selben Stelle x ausgwertet werden, spricht man von einer \textbf{gewöhnlichen Differentialgleichung}.
\end{Definition}

\begin{Definition}{Ordnung, linear, homogen}{}
    Die \textbf{Ordnung} einer Gleichung ist die höchste vorkommende Ableitung. Bsp: $y''$ $\implies$ 2.\\
    
    Eine Differentialgleichung heisst \textbf{linear}, falls jeder Term $y$, $y'$, $y''$ usw. nur linear vorkommt. Wichtig: Falls $y_1(x)$ und $y_2(x)$ Lösungen der selben Differentialgleichung sind, dass ist auch die Linearkombination $ay_1(x) + by_2(x)$.\\
    
    Eine Differentialgleichung heisst \textbf{homogen}, falls keine Terme vorkommen, die rein von den Funktionsvariablen abhängen (also kein $y$, $y''$, ... enthalten). Sonst heisst die Gleichung \textbf{inhomogen}.
\end{Definition}

\begin{Definition}{Anfangswertproblem}{}
    Ein \textbf{Anfangswertproblem} $n$-ter Ordnung ist eine gewöhnliche Differentialgleichung $n$-ter Ordnung zusammen mit $n$ Anfangsbedingungen.
\end{Definition}

\begin{Satz}{Grundprinzip für lineare, inhomogene Differenzialgleichungen}{}
    Die allgemeine Lösung einer linearen, inhomogenen Differentialgleichung hat die Form
    \[
        \underbrace{y(x)}_{\text{Gesamtlösung}}
        = \underbrace{y_{hom}(x)}_{\text{allg. homogene Lösung}}
        + \underbrace{y_p(x)}_{\text{partikuläre Lösung}}
    \]
    
\end{Satz}


\begin{Rezept}{Reduktion der Ordnung}{}
Reduktion einer ODE mit Ordnung $k$ auf Ordnung 1:
Man definiere 
	\begin{equation*}
	f: \mathbb{R} \rightarrow \mathbb{R}^k \qquad x \longmapsto \begin{pmatrix} y(x) \\ y'(x)\\ \cdots \\ y^{(k)}(x)\end{pmatrix} \qquad
	f'(x)= \begin{pmatrix} y'(x) \\ y''(x) \\ \cdots \\ \small{\text{original equation solved for $y^{(k)}(m)$}}
	\end{pmatrix}
	\end{equation*}

Für $k=3$ und $ \rom{1}: y''' = cy'' + by' + ay$:
	\begin{equation*}
	f'(x)= \underbrace{\begin{pmatrix}
	0 & 1 & 0 \\ 0 & 0 & 1 \\ a & b & c
	\end{pmatrix}}_{A(x)} \cdot \begin{pmatrix} y(x) \\ y'(x) \\ y''(x) 
	\end{pmatrix} \qquad \rom{1}': f'(x) = A(x) \cdot f(x)
	\end{equation*}
\end{Rezept}

\begin{Rezept}{Separation der Variablen}{}
	Diese Methode eignet sich für \textbf{Differentialgleichungen erster Ordnung} und ist die einfachste Methode.
	
	Für eine Differentialgleichung der Form
	\begin{equation*}
	y' =\frac{dy}{dx}= h(x)\cdot g(y), \qquad \text{mit } g(y)\neq 0
	\end{equation*}
	gehen wir folgendermassen vor:
	\begin{enumerate}
		\item Wir nehmen alle Teile mit $x$ und alle mit $y$ auf verschiedene Seiten.
		\begin{equation*}
		\frac{1}{g(y)} dy=h(x)dx
		\end{equation*}
		\item Nun integrieren wir direkt
		\begin{equation*}
		\int \frac{1}{g(y)}dy = \int h(x) dx
		\end{equation*}
		\item Durch die Unbestimmtheit der Integrale führen wir eine Konstante $C$ ein. Durch Anfangsbedingungen $y(x_0)=y_0$ (in einem Anfangswertproblem) kann diese bestimmt werden.
	\end{enumerate}
\end{Rezept}

\begin{Rezept}{Variation der Konstanten (1. Ordnung)}{}
	Diese Methode eignet sich für \textbf{inhomogene Differentialgleichungen erster Ordnung}, der Form
	
	\begin{equation*}
	y' = h(x) y + b(x).
	\end{equation*}
	
	Wir benutzen den Grundsatz für inhomogene Differentialgleichungen. D.h. wir suchen die allgemeine homogene Lösung und eine partikuläre Lösung und addieren diese für die gesamte Lösung. Eine partikuläre Lösung kann manchmal erraten werden, ansonsten nutzen wir folgendes Rezept:
	
	\begin{enumerate}
		\item Die homogene Lösung $y_{hom}(x)$ suchen wir mit der Methode \textit{Separation der Variablen}.
		\item Die Integrationskonstante aus Schritt I fassen wir als eine von $x$ abhängige Funktion auf
		\begin{equation*}
		C \rightarrow C(x)
		\end{equation*}
		\item Die entstandene Funktion $y_p(x)$ (homogene Lösung mit $C(x)$ anstelle von $C$) setzen wir als Ansatz in die Differentialgleichung ein und lösen nach $C(x)$ auf. Dies gibt uns die partikuläre Lösung.
		\item  Wir nutzen den Grundsatz für die gesamte Lösung
		\begin{equation*}
		y(x) = y_h(x) + y_p(x).
		\end{equation*}.
	\end{enumerate}
\end{Rezept}

\begin{Rezept}{Euler-Ansatz}{}
	Für lineare, homogene Differentialgleichung $n-ter$ Ordnung mit konstanten Koeffizienten, können wir den \textbf{Euler-Ansatz} verwenden. Wir haben eine Gleichung der Form
	\begin{equation*}
	a_n y^{(n)} + a_{n-1} y^{(n-1)} + \dots + a_0 y = 0,
	\end{equation*}
	wobei $a_0,\dots, a_n \in \R$ und $a_n\neq 0$ sind.
	
	Wir wenden folgendes Rezept an:
	\begin{enumerate}
		\item Setze den \textbf{Euler-Ansatz} $y(x) = e^{\lambda x}$, $\lambda \in\mathbb{C}$ in die Differentialgleichung ein und berechne das \textbf{charakteristische Polynom}.
		\item Finde die Nullstellen $\lambda_k$ mit Vielfachheiten $m_k$ des charakteristischen Polynoms und konstruiere daraus die linear unabhängigen Lösungen gemäss
		\begin{equation*}
		e^{\lambda x}, x \cdot e^{\lambda x}, x^2 \cdot e^{\lambda x}, \dots, x^{m-1} \cdot e^{\lambda x}.
		\end{equation*}
		\item Diese Lösungen bilden ein \textbf{Fundamentalsystem} der Differentialgleichung.
		\item Die allgemeinste Lösung ist eine Linearkombination aller Lösungen im Fundamentalsystem. Die Koeffizienten sind durch die Anfangsbedingungen zu bestimmen.
	\end{enumerate}
\end{Rezept}

\begin{Rezept}{Ersetzen von komplexen Lösungen}{}
%	Richtung: Lösungen der Charakteristischen Gleichung $\lambda_i$ $\to$ Ansatz der \textbf{homogenen} Lösung $y_h$.\\

    %\begin{tabular}{c|c}
     %    \textbf{Lösung für $\lambda$} & \textbf{Linearkombinationen für $y_h(x)$}\\
      %   $\alpha$ & ${\color{green}c_1} \cdot e^{\alpha x}$\\
       %  $\alpha_1 = \alpha_2 = \ldots = \alpha_k$ & ${\color{green}c_1} \cdot e^{\alpha x} + {%\color{blue} x} \cdot e^{\alpha x} + .. + {\color{green}c_k} \cdot {\color{blue} x^{k-1}} \cdot e^{\alpha x}$\\
         %$\alpha + \beta i, \; \beta>0$ & ${\color{green}c_1} \cdot e^{\alpha x} \cdot \sin(\beta x)$\\
         %$\alpha + \beta i, \; \beta<0$ & ${\color{green}c_1} \cdot e^{\alpha x} \cdot \cos(\beta %x)$\\
         %$\alpha_1 + \beta_1 i = .. = \alpha_k + \beta_k i \;\; \beta>0$ & ${\color{green}c_1}e^{\alpha x} \cdot \sin(\beta x) + .. + {\color{green}c_k}{\color{blue} x^{k-1}} \cdot e^{\alpha x} \cdot \sin(\beta x)$\\
         %$\alpha_1 + \beta_1 i = .. = \alpha_k + \beta_k i \;\; \beta<0$ & ${\color{green}c_1}e^{\alpha x} \cdot \cos(\beta x) + .. + {\color{green}c_k} {\color{blue} x^{k-1}} \cdot e^{\alpha x} \cdot \cos(\beta x)$\\
    %\end{tabular}
    Wenn alle Koeffizienten reel sind und $\alpha = \beta + i\gamma$:\\
    \begin{tabular}{c|c}
    $f_1(x) = e^{\alpha x}$ & $\Tilde{f_1}(x) = e^{\beta x} cos(\gamma x)$\\
    $f_2(x) = e^{\overline{\alpha} x}$ & $\Tilde{f_2}(x) = e^{\beta x} sin(\gamma x) $\\
    \end{tabular}
\end{Rezept}

\begin{Rezept}{Mehrfach vorkommende Nullstellen (lineare ODE mit konstanten Koeffizienten)}{}
$\alpha$ Nullstelle mit Vielfachheit $j$. Dann sind folgende Faktoren Teil der Basis der Lösungen:\\
$f_{\alpha, 0}(x) = e^{\alpha x}, \qquad f_{\alpha, 1}(x) = xe^{\alpha x}, \qquad \cdots, \qquad  f_{\alpha, j-1}(x) = x^{j-1} e^{\alpha x} $
\end{Rezept}

\begin{Diverses}{}{}
	Manchmal sind die Terme in $y(x) = Ae^{ix} + Be^{-ix}$ nicht die geeignetste Form und man möchte lieber reelle Funktionen. Dann nutzt man die Formel 
	\[
		\sin x = \frac{e^{ix} - e^{-ix}}{2i} \quad
    	\cos x = \frac{e^{ix} + e^{-ix}}{2}
    \]
	Damit lassen sich zwei neue Integrationskonstanten definieren: \[\widetilde{A} \sin x + \widetilde{B} \cos x\]
	
	\textbf{Rezept:} Für kompliziertere Ausdrücke funktioniert:
	\[
	    y(x) = Ae^{2ix} + Be^{-2ix} = \widetilde{A} \sin (2x) + \widetilde{B} \cos (2x)
	\]
	\textbf{Trick:} $y' + ay = b_1 + b_2 \Longrightarrow f_0 = f_1 + f_2$
\end{Diverses}

\begin{Rezept}{Variation der Konstanten (2. Ordnung)}{}
	Wir suchen die Lösung für eine Differentialgleichung der Form
	\begin{equation*}
	y'' + a_1 y' + a_0 y = g(x)
	\end{equation*}
	\begin{enumerate}
		\item Die homogene Lösung $y_h(x)$ finden wir mittels \textbf{Euler-Ansatz}.
		\item Wir suchen nun eine Lösung der Form $y_p(x) = C_1(x) y_1(x) + C_2(x) y_2(x)$, wobei $y_1(x)$ und $y_2(x)$ aus dem Euler-Ansatz stammen und das System
		\begin{equation*}
		\begin{cases} C_1'(x)y_1(x) + C_2'(x)y_2(x)=0\\C_1'(x)y_1'(x) + C_2'(x)y_2'(x)=g(x)\end{cases}
		\end{equation*}
		d.h.
		\begin{equation*}
		\underbrace{\begin{pmatrix}
			y_1(x) &y_2(x)\\y_1'(x) &y_2'(x)
			\end{pmatrix}}_{=:A} \cdot \vekk{C_1'(x)}{C_2'(x)} = \vekk{0}{g(x)}
		\end{equation*}
		erfüllt sein muss. Wir prüfen also, ob die Determinante
		\begin{equation*}
		\det A = y_1(x)y_2'(x)-y_2(x)y_1'(x)
		\end{equation*}
		nicht verschwindet, denn aus LinAlg wissen wir, dass genau dann eine eindeutige Lösung existiert.
		\item  Wir finden nun $C_1$ und $C_2$ und somit $y_p(x)$ entweder durch
		\begin{enumerate}[(a)]
			\item Inversion der Matrix \begin{equation*}
			\vekk{C_1'(x)}{C_2'(x)} = \frac{1}{y_1(x)y_2'(x)-y_2(x)y_1'(x)} \begin{pmatrix}
			y_2'(x) & -y_2(x)\\-y_1'(x) &y_1(x)
			\end{pmatrix}\cdot\vekk{0}{g(x)}
			\end{equation*}
			und anschliessender Integration $C_1 = \int C_1'(x) dx$, $C_2 = \int C_2'(x) dx$ 
			\item oder mit der direkten Formel
			\begin{align*}
			y_p(x) = &-y_1(x) \int \frac{y_2(x) g(x)}{y_1(x)y_2'(x)-y_2(x)y_1'(x)} dx\\&+ y_2(x) \int \frac{y_1(x) g(x)}{y_1(x)y_2'(x)-y_2(x)y_1'(x)}dx.
			\end{align*}
		\end{enumerate}
		\item Die gesamte Lösung erhalten wir aus $y(x) = y_h(x) + y_p(x)$.
	\end{enumerate}
	
	\textbf{Nützliches:} Bei unchilligen Formeln kann $x\mapsto e^t, x^2\mapsto e^{2t},...$ substituiert werden und anschliessend durch ein Gleichungssystem aufgelöst werden.
\end{Rezept}

\begin{Rezept}{Substitution}{}
	\textbf{Gleichungen der Form} $\boxed{y'=h\left(\frac{y}{x}\right)}$\\
	Substitution
	\begin{equation*}
	z(x)=\frac{y(x)}{x} \qquad \Leftrightarrow \qquad y(x)=x z(x),
	\end{equation*}
	dann wird $y'$ durch
	\begin{equation*}
	y'=z+xz'
	\end{equation*}
	ersetzt.\\
	
	\textbf{Gleichungen der Form} $\boxed{y'=h(ax+by+c)}$\\
	Substitution
	\begin{equation*}
	z(x)=ax+by(x)+c \qquad \Leftrightarrow \qquad y=\frac{z-ax-c}{b},
	\end{equation*}
	dann wird $y'$ durch
	\begin{equation*}
	y'=\frac{z'-a}{b}
	\end{equation*}
	ersetzt.\\
	
	\textbf{Gleichungen der Form} $\boxed{y'=h\left(\frac{ax + by + c}{dx + ey + f}\right)}$\\
	Wir wollen $y(x)$ und $x$ ersetzen. Wir lösen das Gleichungssystem
	\begin{equation*}
	\begin{cases}
	ax+by+c=0\\dx+ey+f=0
	\end{cases}
	\end{equation*}
	und wollen eine eindeutige Lösung $(x_0,y_0)$, d.h. wir fordern
	\begin{equation*}
	\det \begin{pmatrix}a &b \\d &e\end{pmatrix}\neq 0.
	\end{equation*}
	Jetzt setzen wir \begin{equation*}
	z=y-y_0 \qquad \text{und} \qquad t=x-x_0.
	\end{equation*}
	Dann werden $y'$ und $z'$ zu 
	\begin{equation*}
	y'=\frac{dy}{dx} = \frac{d(z+y_0)}{d(t+x_0)} = \frac{dz}{dt} = z'.
	\end{equation*}\\
	\textbf{Gleichungen der Form} $\boxed{y'=\frac{y}{x}h(xy)}$\\
	Substitution 
	\begin{equation*}
	z(x)=xy(x) \qquad \Leftrightarrow \qquad y=\frac{z(x)}{x},
	\end{equation*}
	dann wird $y'$ durch
	\begin{equation*}
	y'=\frac{xz'-z}{x^2}
	\end{equation*}
	ersetzt.\\
\end{Rezept}

\begin{Beispiel}{Substitution: Funktion}{}
		Sei $xy'=y+x^2$ die DGL. Sei $v$ die neue Variable und die Substitution $v = \frac{y}{x}$ oder $y=vx$.\\
	
	\textbf{Lösungsschritt I:} Löse nach $y$ auf: $y = v \cdot x$.
	
	\textbf{Lösungsschritt II:} Berechne $\frac{dy}{dx} = y'$: $y' = v + x v'$
	
	\textbf{Lösungsschritt III:} Setze $y$ und $y'$ in die DGL ein: $x(v'x+v)=vx+x^2$ und löse die DGL normal nach $v$ auf (hier Separation der Variablen). Man bekommt $v=x+c$.
	
	\textbf{Lösungsschritt IV:} Setze $v$ zurück in die Substution $y=vx$ ein. Man bekommt $y=(x+c)x$.
\end{Beispiel}

\begin{Beispiel}{Substitution: Variable}{}
		Sei $x^2y'' - 3xy' + 5y = 0$ die DGL. Sei $x=e^t$ bzw. $t = \log(x)$ die Variablensubstitution.\\
	
	\textbf{Lösungsschritt I:} Definiere $h(t) = y(e^t)$ und berechne $h'$ und $h''$:
	\begin{align*}
		h(t) &= y(e^t) = y(x)\\
		h'(t) &= y'(e^t) e^t = xy(e^t)\\
		h''(t) &= y''(e^t)e^{2t}+y'(e^t)e^t = x^2y''(x)+xy'(x)
	\end{align*}
	
	\textbf{Lösungsschritt II:} Löse auf, sodass alle Terme in der DGL ($x^2y''$, $xy'$, $y$) durch counterparts in $t$ und $h(t)$ ersetzt werden können. Setze in Diffgleichung ein. Man bekommt
	\[
		h'' - 4h' + 5h = 0
	\]
	
	\textbf{Lösungsschritt III:} Löse die Gleichung normal mit $h(t)$. Anschliessend, ersetze $t=\log(x)$
\end{Beispiel}

\begin{Rezept}{Methode des direkten Ansatzes}{}
	Der Ansatz ist geeignet für lineare, inhomogene DGLs $n$-ter Ordnung mit konstanten Koeffizienten. Also eine DGL der Form
	\begin{equation*}
	a_n y^{(n)} + a_{n-1} y^{(n-1)}+\cdots+a_0y = b(x).
	\end{equation*}
	Die homogene Lösung finden wir mit dem Euler-Ansatz. Für die partikuläre Lösung nutzen wir folgende Idee:
	\begin{center}
		$\boxed{\text{Der Ansatz für } y_p(x) \text{hat dieselbe Form wie der inhomogene Term } b(x).}$
	\end{center}
	\renewcommand{\arraystretch}{1.2}
	\begin{center}
		\begin{tabularx}{\textwidth}{X|X}
			\hline 
			\textbf{Inhomogener Term $b(x)$} & \textbf{Ansatz für $y_p(x)$} \\ 
			\hline
			$a$ (const.) & $b$ (const.)\\
			$a_0 + a_1 x + \ldots + a_m x^m$ & $b_0 + b_1 x + \ldots + b_m x^m$\\
			$a e^{\lambda x}$ & $b e^{\lambda x}$\\
			$P(x) e^{\lambda x}$ & $Q(x) e^{\lambda x}$\\
			$P(x) \sin(mx)$ & $Q(x) \sin(mx) + R(x) \cos(mx)$\\
			$P(x) \cos(mx)$ & $Q(x) \sin(mx) + R(x) \cos(mx)$\\
			$a \sin(mx)$ & $c \sin(mx) + d \cos(mx)$\\
			$a \cos(mx)$ & $c \sin(mx) + d \cos(mx)$\\
			$a \sin(mx) + b \cos(mx)$ & $c \sin(mx) + d \cos(mx)$\\
			$\sum_{i=0}^m b_i x^i$ &	$\sum_{i=0}^m A_i x^i$ \\
			$e^{\alpha x}\sum_{i=0}^m b_i x^i$& $e^{\alpha x} \sum_{i=0}^m A_i x^i$ \\ \hline
			$\sin(\omega x)\sum_{i=0}^m b_i x^i$ \newline
			$+ \cos(\omega x)\sum_{i=0}^m c_i x^i$ & $\sin(\omega x)\sum_{i=0}^m A_i x^i$ \newline $+\cos(\omega x)\sum_{i=0}^m B_i x^i$\\ \hline
			$\sinh(\omega x)\sum_{i=0}^m b_i x^i$ \newline $+\cosh(\omega x)\sum_{i=0}^m c_i x^i$ &  	$\sinh(\omega x)\sum_{i=0}^m A_i x^i $\newline$+ \cosh(\omega x)\sum_{i=0}^m B_i x^i$ \\ \hline
			$e^{\alpha x}\sin(\omega x)\sum_{i=0}^m b_i x^i$\newline$+ e^{\alpha x}\cos(\omega x)\sum_{i=0}^m c_i x^i$ & $e^{\alpha x}\sin(\omega x)\sum_{i=0}^m A_i x^i$\newline$+e^{\alpha x}\cos(\omega x)\sum_{i=0}^m B_i x^i$\\ 
			\hline 
		\end{tabularx} 
	\end{center}
	\textbf{Bemerkung I:} Verschiedene Ansätze können additiv kombiniert werden. So wählt man für $b(x)=5x + \sin(x)e^{3x}$ als Ansatz
	\begin{equation*}
	y_p(x) = \underbrace{Ax + B}_{\text{Teil I}} + \underbrace{(C+\sin x + D \cos x)e^{3x}}_\text{Teil II}.
	\end{equation*}
	\textbf{Bemerkung II:} Wenn ein Teil der für $y_p(x)$ zu wählende Funktion bereits in der Lösung des homogenen Problems vorhanden ist, wird der Ansatz zusätzlich mit $x$ multipliziert. Wenn z.B. die homogene Lösung die Form $y_h(x) = Ax + B$ hat, wählt man anstelle des Ansatzes $y_p(x) = ax+b \Rightarrow y_p(x) = x\cdot(ax+b)$.
\end{Rezept}

\begin{Rezept}{Inhomogene AWP}{}
    Wie im Beispiel ($y' + y = 2\sin(x)$ mit $y(0)=0$) vorgehen:
	\begin{enumerate}
	    \item { Homogene Lösung finden: $(\lambda+1)(e^{\lambda x}) = 0 \Rightarrow \lambda = -1 \Rightarrow y_h = c_1 \cdot e^{-x}$ }
	    \item { Gemäss Störfunktion den Ansatz für $y_p$ wählen und dessen
            Ableitungen finden:
            \begin{align*}
              y_p & = c\cdot \sin(x) + d\cdot \cos(x)\\
              y_p' & = c\cdot \cos(x) - d\cdot \sin(x) 
            \end{align*}
	        }
	    \item { Dies in die ursprüngliche DGL einsetzen, um die Koeffizienten des Ansatzes und damit die partikuläre Lösung zu finden:
	       \[ c \cdot \cos(x) - d\cdot \sin(x) + c\cdot \sin(x) + d\cdot \cos(x) \stackrel{!}{=} 2\sin(x)\] \[ \Leftrightarrow (c+d) \cos(x) + (c-d) \sin(x) \stackrel{!}{=} 2\sin(x) ~ \Longleftrightarrow ~ c=1,\;d=-1 \]
	    }
	    \item{
	        Anfangswertbedingungen nutzen um $c_k$ aus der homogenen Lösung zu finden
	        \[ y(0) = \underbrace{c_1 \cdot e^{-x}}_{y_h(0)} + \underbrace{\sin(x)-\cos(x)}_{y_p(0)} = c_1\cdot1 + 0 - 1 \stackrel{!}{=} 0 \Leftrightarrow c_1 = 1 \]
	    }
	    \item{
	        \[ y(x) = 1 + sin(x) - cos(x) \] 
	   }
	\end{enumerate}
\end{Rezept}

%\subsection{Ordinary Differential Equations}

%\subsection{Linear Differential Equations}

%\subsubsection{Linear Differential Equations of Order 1}

%\subsubsection{Linear Differential Equations with constant coefficients}

%with multiple roots, without multiple roots\\

%rezept von übung
