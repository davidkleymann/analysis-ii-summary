%TODO study extremstellen

\subsection{Extremstellen (Kritische Punkte)}

\begin{Definition}[label=R1]{Kritische und reguläre Punkte}{}
	\textbf{Fall $f: \R^n \to \R$}: Sei 	$f: \Omega \subset \R^n \to \R$ differenzierbar. Ein Punkt $p_0 \in \Omega$ heisst \textbf{kritischer Punkt von $f$}, falls $df(x_0)=0$. Ist der Punkt nicht kritisch, heisst er \textbf{regulär}.\\
	
	\textbf{Fall $f: \R^n \to \R^m$}: Sei $f: \Omega \subset \R^n \to \R^m$ differenzierbar. Wir wissen, dass $df(x_0)$ eine lineare Abbildung ist. Diese lässt sich mit einer $m\times n$-Matrix darstellen. Der \textbf{Rang} der Matrix beschreibt die Dimension des Bildes. Für eine $m \times n$ Matrix $A$ gilt \[\text{Rang}(A) \leq \min(m, n)\] Ein Punkt $p_0 \in \Omega$ heisst \textbf{regulär}, wenn die Matrix $df(x_0)$ einen maximalen Rang besitzt. Sonst heisst $p_0$ kritisch.
	
	Ein kritischer Punkt heisst \textbf{degeneriert}, wenn an dieser Stelle die Hesse-Matrix nicht-singulär ist: \[\det(H_f(p_0))=0\] gilt. Ansonsten heisst der Punkt \textbf{nicht-degeneriert}.
\end{Definition}

\begin{Satz}{Extremwerte in einer Variable}{}
	Sei $f: \Omega \subset \R \to \R$, $f \in C^2$ und $x_0 \in \Omega$ ein kritischer Punkt (d. h. $f'(x_0) = 0$), dann gilt
	\begin{itemize}
		\item $x_0$ ist ein Minimum, falls $f''(x_0) > 0$,
		\item $x_0$ ist ein Maximum, falls $f''(x_0) < 0$,
		\item $x_0$ ist ein Sattelpunkt, falls $f''(x_0) = 0$
	\end{itemize}
\end{Satz}

\begin{Rezept}{Eigenwerte finden}{}
	Charakteristisches Polynom mittels $\det (A-\lambda I)$ berechnen. Nullstellen des Polynoms sind die Eigenwerte.
\end{Rezept}

\begin{Satz}{Eigenwert-Kriterium}{}
	Seien $\lambda_1, ..., \lambda_n$ die Eigenwerte einer reellen, symmetrischen $n \times n$ Matrix $A$. Dann gilt
	\begin{align*}
		A \text{ ist positiv definit} &\iff \text{Alle }\lambda_i > 0,\\
		A \text{ ist positiv semi-definit} &\iff \text{Alle }\lambda_i \geq 0,\\
		A \text{ ist negativ definit} &\iff \text{Alle }\lambda_i < 0,\\
		A \text{ ist negativ semi-definit} &\iff \text{Alle }\lambda_i \leq 0
	\end{align*}
\end{Satz}

\begin{Definition}{Hauptminoren}{}
	Die Hauptminor einer symmetrischen, reellen Matrix $A$ sind die nordwestlichen Unterdeterminanten der Matrix $A$. Sie werden mit $A_i$ bezeichnet, wobei $i$ die Grösse der Teilmatrix $A_i$ ist. 
	
	Sei die Matrix
	\[
		A = \begin{pmatrix}
            a_{11} & a_{12} & \hdots & a_{1n}\\
            a_{21} & a_{22} & \hdots & a_{2n}\\
            \vdots & \vdots & \ddots & \vdots\\
            a_{n1} & a_{n2} & \hdots & a_{nn}
        \end{pmatrix}
	\]
	dann sind die $n$ Hauptminoren $A_1, ..., A_n$ gegeben durch
	\begin{align*}
		A_1 &= \det(a_{11}) = a_{11},\\
		A_2 &= \det\begin{pmatrix}
            a_{11} & a_{12} \\
            a_{21} & a_{22}
            \end{pmatrix},\\
       	A_3 &= \det\begin{pmatrix}
            a_{11} & a_{12} & a_{13} \\
            a_{21} & a_{22}& a_{23}\\
            a_{31} & a_{32}& a_{33}
            \end{pmatrix},\\
            &\vdots\\
        A_n &= \det(A)
    \end{align*}
\end{Definition}

\begin{Satz}{Sylvester-Kriterium}{}
	\textbf{NUR FÜR GROSSE MATRIZEN VERWENDEN.}
	Sind $A_1, ..., A_n$ die Hauptminoren der reellen, symmetrischen $n \times n$ Matrix A, dann gilt:
	\begin{align*}
		A \text{ ist positiv definit} &\iff \text{Alle }A_i > 0,\\
		A \text{ ist negativ definit} &\iff \text{Wechselndes Vorzeichen: }A_1 < 0, A_2 > 0, A_3 < 0,...\\
		A \text{ ist indefinit} &\iff \text{Weder alle }A_i \leq 0 \text{ noch }A_i \geq 0 
	\end{align*}
\end{Satz}

\textbf{Rezept}: 1) Gradient berechnen 2) Gradient $=0$ setzen $\implies$ kritische Punkte. 3) Hessematrix berechnen 4) für jeden kritischen Punkt: Falls die Hessematrix positiv definit ist $\implies$ lokales Minimum; falls Hessematrix negativ definit ist $\implies$ lokales Maximum; sonst Sattelpunkt.\\

\textbf{Rezept für abgegrenzte Skalarfelder}: 1) Skalarfeld nach kritischen Punkten untersuchen 2) Alle Begrenzungen des Skalarfelder einzeln nach kritischen Punkten berechnen. Beispiel Dreieck: alle Seiten parametrisieren und alle Eckpunkte untersuchen.

\begin{Rezept}[label=R1]{Extremwertaufgaben ohne Nebenbedingungen}{}
	Gegeben: $f:\Omega\subset \mathbb{R}^n \to \mathbb{R}$ mit $\Omega$ offen (d.h. ohne Rand) und $f$ von der Klasse $C^2$.\\
	Gesucht: Extremalstellen von $f$ in $\Omega$.\\
	\newline
	\textbf{Lösungsschritt I:}\\
	Finde die kritischen Punkte $\{x_0\} \in \Omega$. D.h. alle Punkte, für die gilt
	\begin{equation*}
	df(x_0)=0
	\end{equation*}
	und $x_0 \in \Omega$.\\
	\textbf{Lösungsschritt II:}\\
	Untersuche die Hesse-Matrix von $f$ in den Punkten $\{x_0\}$ um über die Art der Extremum zu entscheiden.
	\begin{align*}
	\text{H}_f(x_0) \text{ ist positiv definit} \qquad &\Rightarrow \qquad x_0 \text{ ist ein lokales Minimum von} f\\
	\text{H}_f(x_0) \text{ ist negativ definit} \qquad &\Rightarrow \qquad x_0 \text{ ist ein lokales Maximum von} f\\
	\text{H}_f(x_0) \text{ ist indefinit} \qquad &\Rightarrow \qquad x_0 \text{ ist ein Sattelpunkt von} f
	\end{align*}
	Wenn die Hesse-Matrix keine klare Aussage ergibt (degenerierte kritische Punkte), muss man die Funktion in einer Umgebung abschätzen (um ein Max/Min zu zeigen) oder konkrete Gegenbeispiele finden.
\end{Rezept}

\begin{Rezept}[label=R2]{Extremwerteaufgaben mit ``einfachen'' Nebenbedingungen}{}
	Gegeben: $f:\Omega\subset \mathbb{R}^n \to \mathbb{R}$ mit Rand $\partial{\Omega}$ und $f$ von der Klasse $C^2$.\\
	Gesucht: Extremalstellen von $f$ in $\Omega$.\\
	\newline
	\textbf{Lösungsschritt I:}\\
	Wir untersuchen das Innere $\overset{\circ}{\Omega}$ analog zu Rezept \ref{R1}.\\
	\newline
	\textbf{Lösungsschritt II:}\\
	Wir parametrisieren den Rand $\partial \Omega$ durch $\gamma(t)$. Die kritischen Punkte sind dann die Punkte $\gamma(t)$ für die gilt
	\begin{equation*}
	\dv{t}(f\circ \gamma)(t) = 0.
	\end{equation*}
	\newline
	\textbf{Lösungsschritt III:}\\
	Bestimme die Art der kritischen Punkte.
	\begin{itemize}
		\item Variante 1 (nur kleinstes Min/ grösstes Max): \\Werte alle kritischen Punkte auf dem Rand explizit aus. D.h. berechne $f(\gamma(t))$.
		\item Variante 2 (Min/Max/Sattelpunkt):\\
		Die Funktion $f(\gamma(t))$ ist nur von einer Variable abhängig. Bestimme die Art der kritischen Punkte anhand der Kriterien für 1D-Funktionen.
	\end{itemize}
	
	\textbf{Lösungsschritt IV:}\\
	Ist der Rand stückweise parametrisiert durch $\gamma_1,\gamma_2,\dots$, muss die Funktion zusätzlich an allen Anfangs- und Endpunkten der $\gamma_i$  explizit ausgewertet werden.\\
	\newline
	\textit{Bemerkung: Das Wort "'einfach"' bedeutet hier, dass wir eine Parametrisierung für den Rand finden können.}
\end{Rezept}

\begin{Rezept}{Extremwertaufgaben mit Nebenbedingungen}{}
	Gegeben: $f:\Omega\subset \R^n \to \R$ und $g:\Omega \to \R^l$ der Klasse $C^1$.\\
	Gesucht: ein Extremum der Funktion $f$ unter der Nebenbedingung $g=0$.\\
	\newline
	\textbf{Lösungsschritt I:}\\
	Bilde die Lagrange-Funktion
	\begin{equation*}
	L(x_1,\dots,x_n,\lambda) = f(x_1,\dots,x_n) - \lambda_1 g_1(x_1,\dots,x_n) - \dots - \lambda_l g_l(x_1,\dots,x_n),
	\end{equation*}
	wobei $g_1,\dots,g_l$ die $l$ Komponenten von $g$ sind. Oft ist $l=1$.\\
	\newline
	\textbf{Lösungsschritt II:}\\
	Bestimme die kriischen Punkte von L, d.h. löse
	\begin{align*}
	\pdv{L}{x_1} &= \pdv{f}{x_1}-\lambda_1\pdv{g_1}{x_1} - \dots -\lambda_l\pdv{g_l}{x_1} = 0\\
	\vdots\\
	\pdv{L}{x_n} &= \pdv{f}{x_n}-\lambda_1\pdv{g_1}{x_n} - \dots -\lambda_l\pdv{g_l}{x_n} = 0.
	\end{align*}
	
	Löse dieses Gleichungssystem mit den zusätzlichen Gleichungen von $g=0$.\\
	\newline
	\textbf{Lösungsschritt III:}\\
	Die Lösungen $(x_1,\dots,x_n)$ sind die Kandidaten für Extremalstellen von $f$. Untersuche die Hesse-Matrix von L um den Typ zu bestimmen.
	\begin{align*}
	\text{Hess}(L)(x_0,\lambda)\text{ ist positiv definit} \quad &\rightarrow  \quad 
	x_0 \text{ ist ein lokales Minimum von } f \text{ auf } g^{-1}(0),\\
	\text{Hess}(L)(x_0,\lambda)\text{ ist negativ definit} \quad &\rightarrow  \quad x_0 \text{ ist ein lokales Maximum von } f \text{ auf } g^{-1}(0),\\
	\text{Hess}(L)(x_0,\lambda)\text{ ist indefinit} \quad &\rightarrow  \quad x_0 \text{ ist ein Sattelpunkt von } f \text{ auf } g^{-1}(0).
	\end{align*}
	
	Ist man nur an den \textbf{globalen Extremwerten} interessiert, spart man sich diesen Schritt und wertet stattdessen die Funktion $f$ an den Kandidaten aus.\\
	\newline
	\textbf{Lösungsschritt IV:}\\
	Ist man auch an Extremalstellen im Innern $\overset{\circ}{\Omega}$ interessiert, sucht man diese mit dem bereits bekannten Rezept.
\end{Rezept}
