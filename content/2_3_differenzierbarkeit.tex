%TODO study diffbarkeit

\subsection{Differenzierbarkeit}

\begin{Definition}{Differenzierbarkeit}{}
    Sei $\Omega \subset \mathbb{R}$ offen, $f$ : $\Omega \to \mathbb{R}$, $x_0\in\Omega$. $f$ heisst \textbf{differenzierbar an Stelle $x_0$}, falls der Grenzwert
    \[
    \lim_{x\ to x_0} \frac{f(x) - f(x_0)}{x - x_0} =:
    f'(x_0) =:
    \frac{df}{dx}
    \]
    existiert. Wir nennen $f'(x_0)$ die Ableitung (das \textbf{Differential}) von $f$ an der Stelle $x_0$. Eine solche Funktion heisst dann \textbf{differenzierbar auf $\Omega$}, wenn sie an jeder Stelle $x_0 \in \Omega$ differenzierbar ist.
\end{Definition}

$f$ diffbar $\Leftrightarrow$ alle Teil-$f$ sind diffbar.

$f,g$ diffbar $\Rightarrow$ $f+g$, $f \cdot g$, $\frac{f}{g}$, $g \circ f$ diffbar

TODO: vielleicht nummern im skript zu chain rule usw?

TODO: skript proposition 3.4.4

TODO: regeln differenzierbarkeit: addition, multiplikation, division, verkettung (proposition 3.4.9)

TODO: tangent space of $f$

\begin{Definition}{Partielle Differenzierbarkeit}{}
	
\end{Definition}

\textbf{Partielle Differenzierbarkeit}: $f$: $\mathbb{R}^n \rightarrow \mathbb{R}^m$, falls:

\[
    \lim_{h \rightarrow 0} \frac{f(x_0 + h e_i)-f(x_0)}{h} =: \frac{\partial f}{\partial x_i}(x_0)
\]

oder generell für alle $e_i$ zusammengefasst in Richtung $v \in \mathbb{R}^n$

\[
    \lim_{h \rightarrow 0} \frac{f(x_0 + h v)-f(x_0)}{h} =: D_v f(x_0)
\]

Dieser $\lim$ existiert $\Leftrightarrow$ in Richtung $e_i$ an Stelle $x_0$ partiell differenzierbar.\\

\textbf{Totale Differenzierbarkeit}: TODO, hatten wir das im Skript: Ja: Def 3.4.2 und Prop. 3.4.4\\

Prop. 3.4.7!

\textbf{Tangent space}

